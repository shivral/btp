\section{Conclusion}
In conclusion, using face recognition technology for attendance management can greatly improve efficiency and accuracy. By combining various algorithms and tools, such as Dlib, CNN, and IP cameras, the system can detect and recognize faces in real-time and mark the attendance of enrolled students automatically.

Implementing such a system requires careful consideration of various factors, such as hardware and software requirements, data privacy and security, and the specific needs of the application. However, with proper planning and execution, the benefits of such a system can be significant, saving time and reducing errors associated with manual attendance management.

Overall, face recognition technology has a wide range of potential applications, and attendance management is just one of them. As the technology continues to evolve and improve, we can expect to see even more innovative and impactful use cases in the future.
\clearpage
\section{Future Work}
\subsection{Ease of access}
An API (Application Programming Interface) endpoint is a communication channel that enables different software applications to interact with each other. In the context of the attendance project, having an API endpoint for both face detection and recognition methods means that any authorized application or user can request attendance reports for a particular class from anywhere within the university network or the web.

To achieve this, the attendance system should expose a web API with two endpoints, one for face detection and another for face recognition. These endpoints should be secured with proper authentication and authorization mechanisms to ensure that only authorized users can access them.

When a user sends a request to the face detection endpoint, the system should respond with the detected faces and their corresponding coordinates in the image or video stream. Similarly, when a user sends a request to the face recognition endpoint, the system should respond with the list of recognized students and their attendance status.

Having these API endpoints can provide many benefits, including easier access to attendance reports, real-time monitoring of attendance, and seamless integration with other university systems such as student information systems or learning management systems. Additionally, this can help reduce the workload of teachers and administrative staff who would otherwise have to manually collect and process attendance data.

\subsection{Report Generation}
In order to ensure accuracy and transparency in the attendance system, it is important to provide each student with a notification of their attendance status. This can be done by sending a message or email to the student's registered contact information, informing them of their attendance in the class and the dates on which they were marked present.

In cases where there may be false negatives or inaccuracies in the attendance record, the notification provides an opportunity for the student to bring this to the attention of the relevant authorities and have the issue resolved without much hassle.

Providing regular updates and notifications also encourages students to be more engaged and involved in their attendance and academic progress, leading to better academic outcomes. Additionally, it promotes transparency and accountability, which is important in maintaining the trust of both students and faculty in the attendance system.