\section{The Area of Work}This project primarily focuses on face recognition and detection using various algorithms, with a particular emphasis on utilizing different labeling techniques to map students to their assigned seats.
% \blindtext


\section{Problem Addressed}
% \blindtext
The face detection attendance system aims to address the problems associated with traditional attendance systems that rely on manual processes. One of the primary issues with manual attendance systems is that they are time-consuming and prone to errors, which can lead to inaccurate attendance records. Additionally, manual attendance systems are vulnerable to fraudulent practices such as proxy attendance, where someone else marks the attendance on behalf of an absent student or employee. These problems can result in loss of productivity and can hinder decision-making processes that depend on accurate attendance data.

The face detection attendance system solves these problems by automating the attendance process using facial recognition and facial detection technology. It accurately identifies and records the attendance of individuals without requiring any manual intervention, thus reducing the time and resources required for attendance management. The system also eliminates the possibility of proxy attendance, as it requires the physical presence of the individual for attendance to be marked. Overall, the face detection attendance system provides a reliable and efficient solution to attendance management problems.
\section{Existing System}
% \blindtext

\subsection{Physical Signing}
The traditional system of taking attendance in college lectures involves a manual process where students sign their names on a paper or a register to indicate their presence. However, with the advent of technology, attendance management systems have evolved to become more efficient and reliable.
% \blindtext

 \subsection{RFID}
 A more recent system is the RFID (Radio Frequency Identification) attendance system, where students are provided with RFID tags that are detected by sensors installed in the lecture room. The system identifies the individual tags and records the attendance automatically.
%  \subsubsection{Qorking of System 2}
%  We can create subsubsection also. 
\section{Creation of bibliography}
Use bibch1.bib file to save your bib format citations. Use the command \cite{saini2010alternative} for referring to a particular article \cite{imre2006majority}. 





